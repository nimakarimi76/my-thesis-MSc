% !TEX root = ../my-thesis.tex
%
\chapter{Introduction}
\label{sec:intro}
With the rapid expansion of cities and the escalating worries regarding the sustainable use of our natural resources, the quest for achieving transportation systems that are not only efficient but also eco-friendly has garnered significant attention and become an urgent global challenge \cite{PRILLWITZ20111590}. 
Sustainable mobility, often referred to as sustainable transportation, aims to develop and implement strategies that promote the efficient movement of people and goods while minimizing negative environmental and social impacts \cite{su8010029, akerman2000european}. 
% Machine learning, a field of artificial intelligence, has emerged as a promising approach to addressing the complexities inherent in sustainable mobility by leveraging data-driven insights and advanced computational algorithms.

Furthermore, machine learning, a field of artificial intelligence, involves the development and application of computational models that can automatically learn and improve from data without explicit programming \cite{zhou2021machine}. 
By analyzing large and diverse datasets, machine learning algorithms can identify patterns, make predictions, and gain valuable insights that aid decision-making processes. In the context of sustainable mobility, machine learning can provide innovative solutions to optimize transportation systems \cite{wagner2022:urban-sm}, improve accessibility, and minimize environmental impact.

Mobility data has been leveraged to address inquiries such as estimating the demand of bus passengers based
on weather conditions utlizing a deep learning artificial neural network model as a regression model
~\cite{fontes2020deep}, using deep learning methods to discover the link between an individual's movement 
patterns and their personality traits~\cite{kim2018method}, examining large-scale mobile data to anticipate 
human trajectories, aiming to gain insights into human mobility patterns utilizing "DeepSpace" 
(An online deep learning framework)~\cite{ouyang2016deepspace}, how people travel between 
cities~\cite{reades2007cellular}, predicting the location of mobile users~\cite{anagnostopoulos2009predicting}.
\cite{toch2019analyzing} has conducted reviews of studies that examine location traces to extract insights about human mobility patterns.



Optimizing transportation systems through machine learning involves the application of data analytics to enhance operational efficiency, reduce congestion, and improve the overall performance of transportation networks. By processing large volumes of data collected from sources such as traffic sensors, GPS devices, and social media platforms, machine learning algorithms can generate real-time traffic predictions, optimize routing, and dynamically adapt transportation services \cite{anagnostopoulos2009predicting, dogra2022iot}. These advancements lead to improved travel experiences, reduced travel times, and more efficient allocation of resources.

Accessibility is a key aspect of sustainable mobility, aiming to ensure that transportation services are available and equitable for all individuals, regardless of their physical abilities, income levels, or geographic location. Machine learning algorithms can help address accessibility challenges by analyzing data on travel demand, demographics, and infrastructure characteristics \cite{de2021:ml:sustainable-transportation}. This enables the identification of underserved areas, optimization of transit routes, and the design of transportation services that cater to the needs of diverse populations \cite{cappelletti2022machine}.

Environmental sustainability is another critical dimension of sustainable mobility. Transportation accounts for a significant portion of global greenhouse gas emissions and is a major contributor to air pollution. Machine learning can play a crucial role in minimizing the environmental impact of transportation by developing predictive models that estimate emissions, optimize energy consumption, and facilitate the integration of clean and renewable energy sources \cite{castaneda2021:sustainable-transportation}. Additionally, machine learning techniques can aid in the design of eco-routing algorithms, which suggest the most environmentally friendly travel routes based on real-time data.

In summary, the combination of machine learning and sustainable mobility presents a promising framework to address the complex challenges faced by transportation systems. By harnessing the power of machine learning algorithms, transportation stakeholders can optimize operations, improve accessibility, and mitigate environmental impact. As the demand for sustainable and efficient mobility solutions continues to grow, machine learning can provide valuable insights and tools to shape the future of transportation, leading to more sustainable, accessible, and environmentally friendly mobility systems.


\section{Machine Learning}
\label{sec:intro:motivation}

Can computers learn like humans do from experience? Yes, through machine learning (ML).
It enhances system performance by using computational methods to learn 
from experience, mainly in the form of data. The key task in machine learning is developing
algorithms that build models from data. By inputting experience data into these algorithms,
we obtain models capable of predicting outcomes for new observations. In the context of
computer science, machine learning focuses on learning algorithms~\cite{zhou2021machine}.


~\cite{mitchell1997machine} provides a more formal definition: 
``A computer program is said to learn from experience E for some class of tasks T and performance measure P, 
if its performance at tasks in T, as measured by P, improves with experience E.``
The term "model" is broadly used to denote outcomes learned from data. In some literature 
(e.g.~\cite{hand2001principles}), "model" might mean a global outcome, while "pattern" signifies a local outcome.

The procedure of constructing models from data using machine learning algorithms is referred to as learning or training. The data employed during this phase is known as training data, where each sample serves as a training example, and the complete collection of training examples is the training set. The model derived from this process, representing the underlying data rules, is also termed a hypothesis, while the actual underlying rules are considered as facts or ground-truth. Thus, the fundamental goal of machine learning is to identify or approximate the ground-truth rules ~\cite{zhou2021machine}.

A wide range of machine-learning algorithms has been developed to address the diverse data and problem types in various machine-learning scenarios ~\cite{hastie2009elements, murphy2012machine}. These algorithms can be understood as exploring a vast space of potential programs, guided by training experience, to discover the most optimal performance. Variation among these algorithms arises from their methods of representing candidate programs, such as decision trees or mathematical functions, and their strategies for navigating this program space, including optimization techniques and evolutionary search methods~\cite{jordan2015machine}.

\subsection{Different Machine Learning Algorithms}
Machine learning encompasses a wide range of learning types and algorithms. The major categories are supervised learning, unsupervised learning, semi-supervised learning, and reinforcement learning \cite{Alpaydin14}.

Supervised learning algorithms are trained on labeled data, which includes inputs and desired outputs. Common supervised tasks are classification, predicting categorical outputs like spam or not spam,
and regression, for instance predicting continuous values like price. Algorithms learn a mapping from inputs
to outputs by examining many examples. Supervised learning can achieve high 
predictive accuracy but relies on large labeled training sets which can be costly to obtain. 
Popular supervised algorithms include logistic regression, support vector machines, neural networks, decision trees, and random forests.

In contrast, unsupervised learning operates on unlabeled data, finding hidden patterns and intrinsic structure within it. Clustering algorithms are a key example, grouping data points that are similar. Other unsupervised techniques like dimensionality reduction and association rule mining also derive insights from the data itself without external labeling. As labeling is not required, unsupervised methods can more easily scale to new problem domains, but the models may not have clear accuracy measures.

In summary, machine learning employs various approaches to train models that can analyze data, recognize patterns, make predictions, or perform tasks. Supervised learning offers predictive accuracy but requires labeled data. Unsupervised learning can find hidden insights in any data without labeling but has less defined performance measures. Together these technologies enable machine learning systems to perform tasks not explicitly programmed.



\subsection{Overfitting}
\label{sec:overfit}


In supervised machine learning, an issue called overfitting arises, leading to poor generalization from observed data to new, unseen data. This results in a model performing well on the training set but poorly on the testing set. Overfitting occurs because the model becomes too specific to the training data and struggles with variations present in the testing data. Overfitted models tend to memorize noise and details of the training set rather than learning the underlying patterns~\cite{Ying_2019}.

Occam's Razor, or the principle of parsimony, advocates using models that contain only what's necessary for effective modeling. Overfitting occurs when models or procedures include more terms than needed, violating parsimony. Two types of overfitting are identified: using excessively flexible models and incorporating irrelevant components. Overfitting is undesirable due to resource wastage, potential prediction errors, worse decisions in feature selection, degraded predictions, and reduced portability of models. Portable models, adhering to Occam's Razor, are preferred for their broader applicability and reproducibility across locations~\cite{hawkins2004of,cook2016overfitting}.


\subsection{Spliting the Dataset}

In both statistical and machine learning model development, a common practice is to divide the dataset into two parts: training and testing ~\cite{hastie2009elements}. The training set is utilized to estimate the model's unknown parameters, while the model's accuracy is assessed using the testing dataset. This division is essential to prevent overfitting, where a model becomes too tailored to the data, potentially leading to poor predictions in new situations. By reserving a subset of the data for testing, the model's performance can be evaluated before actual deployment, safeguarding against issues arising from overfitting.

The simplest and probably the most common strategy to split such a dataset is to randomly sample a fraction of the dataset. For example, 80\% of the rows of the dataset can be randomly chosen for training, and the remaining 20\% can be used for testing. Various strategies are showcased in ~\cite{roshan2022split} for efficiently and optimally dividing the dataset.

\subsection{Validation}
\label{sec:cv}
Furthermore, it's a frequent practice to reserve a segment of the training set for validation. This validation subset serves purposes such as refining model performance through hyper-parameter or regularization parameter selection (e.g. the number of hidden units—layers and layer widths—in a neural network~\cite{ripley2007pattern}). Validation datasets also can be used for regularization by early stopping (stopping training when the error on the validation data set increases, as this is a sign of over-fitting to the training data set)~\cite{Prechelt2012}.

To ensure greater result stability and maximize the use of valuable data for training, datasets can be repeatedly divided into multiple training and validation subsets. This practice, referred to as cross validation, is employed. Additionally, an independent test dataset held apart from cross validation is typically utilized to validate the model's performance.~\cite{browne2000cv} provides an overview of various cross validation techniques.

\section{Thesis Structure}
\label{sec:intro:structure}

This thesis is organized into four chapters. Evidently, this Chapter provided background information and an introduction to the key concepts of sustainable mobility and macine learhning. 

\textbf{Chapter \ref{sec:data}} presents the data analysis process. This includes converting raw GPS data into a structured CSV format, performing exploratory analysis, and examining correlations between variables.

In \textbf{Chapter \ref{sec:model}}, the methodology for developing machine learning models is discussed. An overview of modeling concepts is provided, followed by explanations of the logistic regression, random forest, and neural network algorithms used. The application of cross validation techniques is also described. 

Finally, \textbf{Chapter \ref{sec:results}} discuss the results obtained from training and evaluating the models. Performance metrics like accuracy and confusion matrices are analyzed. The relative strengths and limitations of the models are reviewed. 

